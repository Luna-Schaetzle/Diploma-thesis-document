% Maby merch with used technologies

\chapter{Used Programming Languages}
\label{chap:used_programming_languages}


\section{Python}


To enhance readability and comprehension, the most widely used libraries are explained in the following sections.

\subsection{MediaPipe}

\subsection{Transformers}

\subsection{JSON}




\textbf{Gabriels Part}

\section{HTML, CSS, and JavaScript in Combination with Vue.js}

In this project, the languages HTML, CSS, and JavaScript were used in conjunction with Vue.js to create an interactive and dynamic user experience. For more information about Vue.js, see Chapter \ref{chap:used_technologies}, Section "Vue.js."

\subsection{HyperText Markup Language (HTML)}

HyperText Markup Language (HTML) is the standard markup language for creating and structuring content on the web. It serves as the backbone of web pages by organizing content through elements represented by tags. Key features of HTML include:

\begin{itemize}
    \item \textbf{Structure Definition:} Tags such as \texttt{<html>}, \texttt{<head>}, and \texttt{<body>} define the structural hierarchy of a web page.
    \item \textbf{Content Organization:} Elements like headings, paragraphs, links, images, and tables provide a clear and user-friendly layout.
    \item \textbf{Web Compatibility:} HTML is universally supported, ensuring seamless integration across browsers and devices.
\end{itemize}

As one of the core technologies of the World Wide Web, alongside CSS and JavaScript, HTML enables the creation of interactive and visually appealing websites. Its simplicity and adaptability make it an essential tool for web development.

\cite{HTML-Wikipedia}

\subsection{Cascading Style Sheets (CSS)}

Cascading Style Sheets (CSS) is a style sheet language designed to control the visual presentation of web pages. CSS enhances the user experience by allowing developers to define the look and feel of a website. Key functionalities of CSS include:

\begin{itemize}
    \item \textbf{Design Customization:} Control over layout, colors, fonts, and spacing for a cohesive visual identity.
    \item \textbf{Responsive Design:} Ensures consistent and optimized appearance across different devices and screen sizes.
    \item \textbf{Cascading Rules:} Allows styles to be applied at element, class, or global levels, offering flexibility in design.
\end{itemize}

As a foundational technology of the web, CSS plays a vital role in creating modern, responsive, and aesthetically pleasing websites.

\cite{CSS-Wikipedia}

\subsection{JavaScript}

JavaScript is a high-level programming language used to add interactivity and dynamic content to web pages. It works seamlessly alongside HTML and CSS to create rich and engaging user experiences. Key features of JavaScript include:

\begin{itemize}
    \item \textbf{Dynamic Content:} Enables animations, form validation, and real-time updates.
    \item \textbf{Client and Server-Side Usage:} Runs in web browsers via JavaScript engines and supports server-side applications through platforms like Node.js.
    \item \textbf{Extensive Ecosystem:} Offers libraries, frameworks, and tools for building feature-rich web applications.
\end{itemize}

JavaScript’s flexibility and versatility have established it as a cornerstone of web development, making it essential for developing interactive and responsive applications.

\cite{JavaScript-Wikipedia}

\subsection{Vue.js}

write about that VueJS is and that it uses those languages

\section{Type Script}

\subsection{Axios}

\textbf{Flos Part}
