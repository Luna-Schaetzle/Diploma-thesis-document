% Maybe  merch with used technologies

\chapter{Used Programming Languages}
\label{chap:used_programming_languages}
\textbf{Author:} Luna Schätzle

\textbf{Author:} Florian Prandstetter ~\ref{sec:TypeScript}


\section{Python}

Python is a high-level, interpreted programming language renowned for its simplicity, readability, and versatility. It supports multiple programming paradigms, including procedural, object-oriented, and functional programming. Python's design philosophy emphasizes code readability, notably using significant indentation, which enhances the clarity and maintainability of its codebase. Its comprehensive standard library and supportive community have contributed to its widespread adoption across various domains, such as web development, data science, machine learning, and automation.

Key features of Python include:

\begin{itemize}
    \item \textbf{Easy to Code:} Python's syntax is clear and concise, making it accessible to both beginners and experienced programmers.
    \item \textbf{Free and Open Source:} Python is freely available for use and distribution, including for commercial purposes, under an open-source license.
    \item \textbf{Object-Oriented Language:} Python supports object-oriented programming paradigms, facilitating code reuse and modularity.
    \item \textbf{GUI Programming Support:} Python offers libraries like Tkinter, PyQt, and wxPython for developing graphical user interfaces.
    \item \textbf{High-Level Language:} As a high-level language, Python abstracts complex details of the computer's hardware, allowing developers to focus on coding solutions without worrying about memory management.
\end{itemize}

For a comprehensive understanding of Python's features and capabilities, refer to the official Python documentation \cite{python-intro-w3schools}.

To enhance readability and comprehension, the most widely used libraries are explained in the following sections.

\subsection{Transformers}

The \texttt{Transformers} library, developed by Hugging Face, is an open-source Python package that provides implementations of state-of-the-art transformer models. 
It supports multiple deep learning frameworks, including PyTorch, TensorFlow, and JAX, facilitating seamless integration across various platforms. 
The library offers access to a vast array of pre-trained models tailored for tasks such as natural language processing, computer vision, 
and audio analysis. Utilizing these pre-trained models enables researchers and practitioners to achieve high performance in tasks like text classification, 
named entity recognition, and question answering without the necessity of training models from scratch, thereby conserving computational resources and time.

\cite{transformers}

\subsection{JSON}

JavaScript Object Notation (JSON) is a lightweight, text-based data interchange format that is easy for humans to read and write, and straightforward for machines to parse and generate. In Python, the built-in \texttt{json} module provides functionalities to serialize Python objects into JSON-formatted strings and deserialize JSON strings back into Python objects. This module supports the conversion of fundamental Python data types, such as dictionaries, lists, strings, integers, and floats, into their corresponding JSON representations. The \texttt{json} module is indispensable for tasks involving data exchange between Python applications and external systems, particularly in web development contexts where JSON is a prevalent format for client-server communication 

\cite{python-json}.


\section{HTML, CSS, and JavaScript in Combination with Vue.js}

In this project, the languages HTML, CSS, and JavaScript were used in conjunction with Vue.js to create an interactive and dynamic user experience.

\subsection{HyperText Markup Language (HTML)}

HyperText Markup Language (HTML) is the standard markup language for creating and structuring content on the web. It serves as the backbone of web pages by organizing content through elements represented by tags. Key features of HTML include:

\begin{itemize}
    \item \textbf{Structure Definition:} Tags such as \texttt{<html>}, \texttt{<head>}, and \texttt{<body>} define the structural hierarchy of a web page.
    \item \textbf{Content Organization:} Elements like headings, paragraphs, links, images, and tables provide a clear and user-friendly layout.
    \item \textbf{Web Compatibility:} HTML is universally supported, ensuring seamless integration across browsers and devices.
\end{itemize}

As one of the core technologies of the World Wide Web, alongside CSS and JavaScript, HTML enables the creation of interactive and visually appealing websites. Its simplicity and adaptability make it an essential tool for web development.

\cite{HTML-Wikipedia}

\subsection{Cascading Style Sheets (CSS)}

Cascading Style Sheets (CSS) is a style sheet language designed to control the visual presentation of web pages. CSS enhances the user experience by allowing developers to define the look and feel of a website. Key functionalities of CSS include:

\begin{itemize}
    \item \textbf{Design Customization:} Control over layout, colors, fonts, and spacing for a cohesive visual identity.
    \item \textbf{Responsive Design:} Ensures consistent and optimized appearance across different devices and screen sizes.
    \item \textbf{Cascading Rules:} Allows styles to be applied at element, class, or global levels, offering flexibility in design.
\end{itemize}

As a foundational technology of the web, CSS plays a vital role in creating modern, responsive, and aesthetically pleasing websites.

\cite{css-introduction-w3schools}


\subsection{JavaScript}

JavaScript is a high-level programming language used to add interactivity and dynamic content to web pages. It works seamlessly alongside HTML and CSS to create rich and engaging user experiences. Key features of JavaScript include:

\begin{itemize}
    \item \textbf{Dynamic Content:} Enables animations, form validation, and real-time updates.
    \item \textbf{Client and Server-Side Usage:} Runs in web browsers via JavaScript engines and supports server-side applications through platforms like Node.js.
    \item \textbf{Extensive Ecosystem:} Offers libraries, frameworks, and tools for building feature-rich web applications.
\end{itemize}

JavaScript’s flexibility and versatility have established it as a cornerstone of web development, making it essential for developing interactive and responsive applications. 

\cite{JavaScript-introduction-w3schools}


\subsection{Vue.js}

Vue.js is a progressive, open-source JavaScript framework that provides a robust foundation for the development of sophisticated user interfaces and single-page applications. By leveraging standard web technologies—namely HTML, CSS, and JavaScript—Vue.js streamlines the development process through its intuitive and adaptable architecture. Its key attributes include:

\begin{itemize}
    \item \textbf{Component-Based Architecture:} Facilitates modular development by enabling the creation of reusable components that encapsulate both data and functionality.
    \item \textbf{Reactivity System:} Automatically synchronizes the Document Object Model (DOM) with underlying data changes, ensuring a dynamic and responsive user interface.
    \item \textbf{Directives and Template Syntax:} Employs a declarative approach to simplify data binding and event handling.
    \item \textbf{Vue CLI:} Offers a comprehensive command-line interface for project scaffolding, build configuration, and plugin management.
    \item \textbf{Vue Router and Vuex:} Integrates official libraries for client-side routing and state management, thereby enhancing scalability and maintainability.
    \item \textbf{Vibrant Community Support:} Benefits from an active ecosystem that provides extensive documentation, tutorials, and a wide range of third-party plugins.
    \item \textbf{Performance Optimization:} Utilizes a Virtual DOM and efficient rendering algorithms to optimize performance, even in complex applications.
    \item \textbf{Enhanced Developer Experience:} Incorporates developer-centric tools such as Vue Devtools and an intuitive CLI to boost productivity and simplify debugging.
\end{itemize}

In summary, Vue.js equips developers with the necessary tools to efficiently construct interactive and feature-rich web applications, making it a widely adopted framework in contemporary web development 

\cite{vuejs_docs}

\section{Type Script}
\label{sec:TypeScript}

TypeScript is a superset of JavaScript that adds static type definitions to the language. It is designed for the development of large-scale applications and transcompiles to JavaScript.
JavaScript is not very strict when it comes to types. This can lead to issues during development. Typescript adds a type system to JavaScript, which can help to catch errors early in the development process.

%\cite{ts-introdution}


\subsection{Axios}
Axios is the key library used to make HTTP request from the frontend to the backend. It is a promise-based HTTP client for the browser and Node.js. It is used to make asynchronous requests to a server, and it returns a promise that resolves with the response data. 

\cite{axios_docs}

