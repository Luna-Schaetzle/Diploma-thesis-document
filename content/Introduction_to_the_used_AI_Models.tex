\chapter{Introduction to the used AI Models}
\label{cha:Introduction_to_the_used_AI_Models}

For the Diploma thesis, there are many different AI models that are in use. There are different Types of AI models, such as:
\begin{itemize}
    \item LLMs (Large Language Models)
    \item Defusion Models (Models that are used to create images)
    \item Object Detection Models (Models that are used to detect objects in images)
    \item Face Recognition Models (Models that are used to recognize faces in images)
\end{itemize}

In the following chapters, the different Types and the used models will be explained in more detail.

\section{LLMs}

-- Insert explanation of LLMs here --

source: https://www.ibm.com/topics/large-language-models

\section{Utilized LLMs}

In the context of this diploma thesis, various free and commercial large language models (LLMs) 
were evaluated to determine their suitability for integration. Leveraging the Ollama application, 
we were able to test and compare several LLMs. Additionally, we explored different ChatGPT models available through the OpenAI API.
OpenAI offers a range of models that vary in terms of size and complexity, with more advanced models incurring higher usage costs.

\cite{OpenAI_API_overview}
\cite{WhatisOllama}

\section{Models Accessed via Ollama}

For this project, we tested multiple models available through the Ollama application, 
which can be obtained via the Ollama download server.

Since Ollama operates locally, the selected models needed to meet specific requirements to ensure efficient performance. 
Consequently, we evaluated models of varying sizes and levels of complexity.

\subsection{Model Selection Criteria}

The selection of models was guided by the following criteria:

\begin{itemize}
\item \textbf{Model Size:} The model must be capable of running on the server without exceeding available memory capacity.
\item \textbf{Performance Speed:} The response time of the model, i.e., how quickly it can generate output.
\item \textbf{Complexity:} The model's ability to handle complex prompts and generate coherent, contextually accurate text.
\item \textbf{Accuracy:} The overall precision of the model's responses, particularly in terms of factual correctness and linguistic quality.
\item \textbf{Language Support:} The model's proficiency in understanding and generating text in multiple languages, particularly English and German.
\item \textbf{User Experience:} The model's overall usability and user-friendliness, including ease of integration and customization.
\end{itemize}

There is often a trade-off between these criteria. 
Larger models tend to exhibit higher accuracy and greater contextual understanding but are generally slower and 
require more computational resources.

\subsection{Challenges in Model Testing and Updates}

One significant challenge lies in the rapid development and frequent release of new models, 
which complicates the process of continuous integration and comprehensive evaluation of recent advancements. 
Regular testing and updates are imperative to ensure the incorporation of state-of-the-art models while maintaining system reliability and relevance.

Another challenge involves achieving an optimal balance between performance, accuracy, and resource efficiency, 
ensuring that the chosen model meets the application’s functional requirements without compromising the overall user experience.

During the evaluation process, we encountered several obstacles. For instance, 
identifying standardized questions that could be uniformly answered by all models proved challenging. 
Some smaller models demonstrated limitations in addressing certain questions comprehensively. Additionally, 
specialized models, while excelling in niche areas, often lacked the ability to provide detailed answers across a broader range of topics.

For the final evaluation phase, we employed a diverse question set consisting of self-crafted questions, 
publicly available questions from online sources, and queries generated by ChatGPT-4 to ensure a comprehensive assessment covering a wide spectrum of queries.

\subsection{Model Selection for the Final Application}

In the final implementation, users are provided with a curated list of recommended models from which they can select their preferred option. 
This list was carefully compiled based on our comprehensive testing and reflects the models that demonstrated 
the best balance between performance, accuracy, and resource efficiency.

For the production version of the application, 
this list must be updated periodically to include newly released models and maintain optimal performance.

\subsection{Models Evaluated During Testing}

A detailed comparison of the models tested during the evaluation phase is provided in the following sections, 
highlighting their respective strengths and limitations.

\subsection{Models Integrated into the Final Application}

The final selection of models integrated into the application reflects the outcomes of our performance benchmarks and user-centric assessments, 
ensuring a robust and adaptable solution for end-users.


\section{Ollama Model Testing and Evaluation}

Based on the aforementioned criteria, we conducted an extensive evaluation of the following models:

\subsection{Quantitative Evaluation Methods}

For the quantitative evaluation, we focused on key performance metrics to assess the efficiency and reliability of each model:

\begin{itemize}
    \item \textbf{Response Time:} The time taken by the model to generate a response after receiving input.
    \item \textbf{CPU Usage:} The percentage of CPU resources utilized during model execution.
    \item \textbf{GPU Usage:} The extent to which GPU resources were leveraged to enhance performance.
    \item \textbf{Memory Usage:} The amount of RAM consumed while the model was running.
    \item \textbf{Multiple Choice Question Answering:} The accuracy of the model when answering structured multiple-choice questions.
    \item \textbf{Translation Quality:} Measured using the BLEU (Bilingual Evaluation Understudy) score, which evaluates the similarity between the model-generated translation and a human reference translation. \cite{BLUE-Score-Wikipedia}
    \item \textbf{Text Generation Quality:} Assessed using the ROUGE (Recall-Oriented Understudy for Gisting Evaluation) score to measure the overlap between the generated text and reference texts.
\end{itemize}

\cite{ROUGE-BLUE-Score}


\subsection{Qualitative Evaluation Methods}

Qualitative evaluation requires human judgment and is inherently resource-intensive. Therefore, our primary focus was placed on the quantitative evaluation. Nevertheless, we conducted qualitative assessments for specific criteria where human input was indispensable:

\begin{itemize}
    \item \textbf{Contextual Understanding:} The model's ability to interpret and respond accurately to the context of the input text.
    \item \textbf{Factual Correctness:} The accuracy of the factual information provided in the model's output.
    \item \textbf{Linguistic Quality:} The grammatical correctness, fluency, and coherence of the generated text.
    \item \textbf{Emotional Intelligence:} The extent to which the model's output mimics human-like empathy, tone, and emotional nuance.
\end{itemize}

By combining both quantitative and qualitative evaluation methods, 
we obtained a comprehensive understanding of each model's strengths and limitations.

%############################################

\subsection{Models which were tested}

list all the tested models and why they were tested

\subsection{Data Collection}

Explain how the data was collected and what data was collected.
with the Python Skripts / Implement a Python Skript that collects the data as an example 
Maby write that you used ChatGPT to get a good Skript for the data collection
Explain how the qualitative data was collected and what data was collected.

\subsection{Testresults and Analysis}

Explain how the data was analyzed and interpreted.
implement nice Graphics

\subsection{Model Comparison for different Use Cases and Scenarios}

Explain wich modell is best for what

\subsection{Model Selectied for the Final Application}

Explain why the selected model was selected and what the criteria were








