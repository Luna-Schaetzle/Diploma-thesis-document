\chapter{Server Infrastructure}
\label{chap:Server_Infrastructure}
\textbf{Author:} Florian Prandstetter

\section{Introduction}

The server infrastructure is the backbone of every IT system. It is responsible for hosting the applications and services that are used by the clients.
A well-designed server infrastructure can significantly improve the server performance. It also ensures that the server is reliable and scalable.

In this chapter, the different components of a server infrastructure will be discussed. The importance of each component will be explained.

\section{Server Infrastructure Components}

The main components of the projects server infrastructure are the following:

\subsection{Server Hardware}

The server hardware consists of the physical components that make up the server. 
It is explained in more detail in Chapter \ref{chap:Server_Hardware}.

\subsection{Networking}

The networking components of a server infrastructure are responsible for connecting the server to the internet and other devices.
This includes routers, switches, and firewalls.

While networking is a broad topic, it is essential to have a basic understanding of how it works to build a server infrastructure.
Security is also an important aspect of networking. Firewalls and other security measures are used to protect the server from cyber attacks.   

To ensure security and reliability, the server in this project is connected to a secure network. The network is protected by a firewall and other security measures to prevent unauthorized access.

\cite{networking}

\subsection{Remote Management}

Remote management tools are used to monitor and manage the server from a remote location.
This is essential for maintaining the server and ensuring that it is running smoothly.

There are different remote management tools available, such as SSH, RDP, and VNC. 

The server in this project is managed using SSH. This allows the server administrator to access the server remotely and perform maintenance tasks.
To access the server from a different network, a VPN connection is used. This ensures that the connection is secure and encrypted.
The tool used for the VPN connection is Tailscale.

\cite{remote_management}
\cite{Tailscale}

\subsection{Backup and Recovery}

Regular backups are essential to ensure that data is not lost in case of a hardware failure or other disaster.
There are different backup methods, such as full backups, incremental backups, and differential backups.
It is important to have a backup strategy in place to ensure that data can be recovered quickly and efficiently.

The backups for the server in this project are stored on an external hard drive. This hard drive is connected to the server and automatically backs up the data at regular intervals.

\subsection{Containerization}

Containerization is a method of packaging and deploying applications in a lightweight, isolated environment called a container.
Containers are portable and can run on any system that supports containerization.
This makes it easy to deploy applications and services across different environments.

The server in this project uses Docker for containerization. Docker allows the server administrator to package applications and services in containers and deploy them on the server.
This makes it easy to manage and scale the server infrastructure. For a more detailed explanation of Docker, refer to Chapter \ref{sec:docker}.

\cite{containerization} 

\section{Conclusion}

The server infrastructure is a critical component of any IT system. It is responsible for hosting the applications and services that are used by the clients.

By understanding the different components of a server infrastructure, server administrators can design and implement a robust and reliable server infrastructure that meets the needs of the organization.

In this project, the server infrastructure is designed to be secure and reliable. By using the right hardware, networking components, and other components, the server infrastructure can provide the necessary resources and services to support the AI Hub and other applications.
