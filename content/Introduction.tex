\chapter{Introduction: AI in the Industry and Education Environment}
\label{chap:introduction}
\textbf{Author:} Luna P. I. Schätzle, Florian Prandstetter, Gabriel Mrkonja

% write something about the current Rise of AI and the difficulties of the implementation in the industry and education environment
% Then split the students and thier main field of study in the Diploma thesis


% Maby write something about the whay we made the documentations

%This chapter explains the rationale behind choosing the topic. It outlines the objectives and scope of the overall project, as well as the technical and economic context in which it is situated.

%#############################################################################################################

In the context of ongoing digital transformation, the use of artificial intelligence (AI) is becoming increasingly critical across various industries. 
AI technologies are employed to automate processes, optimize production, and enhance quality, while in the education sector, they support learning processes, 
personalize educational content, and provide targeted feedback to students. Nonetheless, 
the implementation of AI solutions in both industry and education remains challenging due to the inherent complexity of these technologies, a shortage of specialized expertise, 
and the high costs associated with their development and deployment.

\section{Objectives and Scope}

The primary objective of this diploma thesis is to develop AI-driven solutions that address specific challenges in both industrial and educational settings.

The scope of the project encompasses the following tasks:
\begin{itemize}
    \item \textbf{Task 1:} Establish a server (API) to host and run AI models.
    \item \textbf{Task 2:} Develop a robust backend for the AI Hub and the Code Extension.
    \item \textbf{Task 3:} Research and identify the most suitable AI models for various tasks, and integrate them into the backend of the AI Hub and the Code Extension.
    \item \textbf{Task 4:} Create an AI Hub for students to access a variety of AI tools and resources, including chatbots, image recognition, and natural language processing (NLP) models.
    \item \textbf{Task 5:} Implement an AI solution for coding assistance, available as an extension within a code editor.
    \item \textbf{Task 6:} Develop an AI-driven service on an edge device (e.g., Raspberry Pi) to explore the potential of AI in industrial applications with minimal resources.
    \item \textbf{Task 7:} Compile comprehensive documentation of the project outcomes, covering the conceptual framework, theoretical foundations, practical implementation, primary and alternative solution approaches, results, and their interpretation.
\end{itemize}

\section{Technical and Economic Context}

The project is embedded within the current wave of digital transformation, which has significantly increased the importance of AI technologies in both industrial and educational environments. By leveraging state-of-the-art AI models and tools—such as chatbots, image recognition, and natural language processing—the project aims to develop innovative solutions tailored to specific challenges in these fields.

Furthermore, the project will investigate the feasibility of deploying AI-driven services on edge devices, like Raspberry Pi, to explore cost-effective and resource-efficient industrial applications. The successful execution of the project relies on the team’s combined expertise in software development, AI technologies, and project management.

\section{Team Composition}

The project team consists of three members:
\begin{itemize}
    \item Luna Schaetzle (Project lead)
    \item Florian Prandstetter
    \item Gabriel Mrkonja
\end{itemize}

Each team member is assigned specific roles and responsibilities, which are detailed in the following sections.

\section{Detailed Task Description}

\subsection{Luna Schaetzle}

Luna Schaetzle is responsible for a wide range of tasks and duties, including:
\begin{itemize}
    \item \textbf{Team Leadership and Project Management:} Serving as the team leader, project manager, and project lead, she oversees the planning, coordination, and monitoring of all project activities.
    \item \textbf{Documentation and Repository Management:} Ensuring comprehensive documentation of project outcomes and managing the GitHub repository.
    \item \textbf{Backend Development:} Leading the development of the backend for both the AI Hub and the Code Extension.
    \item \textbf{Website Development:} Directing the design and implementation of the AI Hub's website.
    \item \textbf{Open Source Evaluation:} Assessing the economic implications and potential of open-source technologies.
\end{itemize}

\subsection{Florian Prandstetter}

Florian Prandstetter is in charge of some key tasks to ensure the projects success.
These tasks are:

\begin{itemize}
    \item \textbf{Setting up a server:} Building and Maintaining
    \item \textbf{Managing the Servers Operating System:} 
    \item \textbf{Managing a Docker Service:} 
    \item \textbf{Developing an VS Code extension:} 
    \item \textbf{Analyzing the OS Market:} 
\end{itemize}

\subsection{Gabriel Mrkonja}
% Insert Gabriel Mrkonja's detailed tasks and responsibilities here.


\section{Structure of the Diploma Thesis}

To enhance readability and clarity, the diploma thesis is organized into distinct parts that mirror the various objectives of the project:

\begin{itemize}
    \item \textbf{Part 1: Introduction and Project Overview} — Provides an introduction to the project, outlining its objectives, scope, and technical as well as economic context.
    \item \textbf{Part 2: Hardware Components} — Describes the hardware utilized in the project, including the Raspberry Pi and the server.
    \item \textbf{Part 3: Theoretical Foundations} — Covers the theoretical background, detailing the operating systems and programming languages employed.
    \item \textbf{Part 4: Implementation of Large Language Models} — Details the integration of large language models, encompassing the hosted Flask service, the Student AI Hub, and the VS Code extension.
    \item \textbf{Part 5: Implementation of Object Detection} — Focuses on object detection techniques, presenting both an introduction and a step-by-step account of the implementation process.
    \item \textbf{Part 6: Economic Evaluations} — Examines the economic aspects of AI in education and industry, including evaluations of open-source approaches and their financial implications.
    \item \textbf{Part 7: Conclusion and Proof of Work} — Summarizes the key findings and outcomes of the project while providing an outlook on future developments.
\end{itemize}

Each part is further divided into chapters, sections, and subsections, ensuring a structured and comprehensive overview of the project.
\footnote{Under every chapter, the author of the chapter is listed.}

