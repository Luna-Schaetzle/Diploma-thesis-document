\chapter{Glossary}
\addcontentsline{toc}{chapter}{Glossary}
\markboth{Glossary}{Glossary}
\label{glossary}
\thispagestyle{plain}

\setcounter{page}{1}
\pagenumbering{roman}

This section provides definitions and explanations for all the terms used throughout this document.

\begin{description}[leftmargin=!,labelwidth=\widthof{\bfseries vs}]
    \item[AI (Artificial Intelligence)] A branch of computer science dedicated to the development of systems capable of performing tasks that typically require human intelligence, such as learning, reasoning, and problem-solving.
    \item[API (Application Programming Interface)] A set of protocols, routines, and tools designed to build software applications and enable communication between distinct software components.
    \item[Bibtex (Bibliographic Format in LaTeX)] A reference management tool and bibliographic format used in LaTeX for organizing and formatting lists of literature and references.
    \item[BLEU (Bilingual Evaluation Understudy)] A metric for assessing the quality of machine-translated text by comparing it against one or more human-generated reference translations.
    \item[CPU (Central Processing Unit)] The primary component of a computer that executes instructions and performs arithmetic, logic, and control operations.
    \item[DE (German)] A language code indicating content or context in the German language.
    \item[Dipl.-Ing. (Diplom-Ingenieur)] An academic engineering degree equivalent to a Master’s level qualification, as used in the context of Albert Greinöcker.
    \item[Dr. (Doctor)] An academic title awarded to individuals who have earned a doctoral degree, as exemplified by Albert Greinöcker and Manuela Schätzle.
    \item[DSG (Data Protection Act)] Legislation governing the protection of personal data, corresponding to the German Datenschutzgesetz.
    \item[DRM (Digital Rights Management)] A set of technologies and strategies employed to control the use, distribution, and modification of digital content after its sale.
    \item[EN (English)] A language code representing the English language.
    \item[EU (European Union)] A political and economic union comprising member states located primarily in Europe.
    \item[GDPR (General Data Protection Regulation)] A comprehensive regulation enacted by the European Union to protect individuals' personal data and privacy.
    \item[GPL (GNU General Public License)] A widely adopted free software license that ensures end users have the freedom to run, study, share, and modify the software.
    \item[GPU (Graphics Processing Unit)] A specialized electronic circuit designed to accelerate the creation and rendering of images by rapidly manipulating and altering memory.
    \item[HTL (Higher Technical Federal College)] An Austrian technical secondary school providing advanced education in technical subjects, as exemplified by the Höhere Technische Bundeslehr- und Versuchsanstalt.
    \item[HW (Hardware)] The physical components of a computer system, including electronic circuits, devices, and machinery.
    \item[JSON (JavaScript Object Notation)] A lightweight data interchange format that is both human-readable and machine-parsable.
    \item[KI (Artificial Intelligence)] The German abbreviation for artificial intelligence, referring to systems capable of exhibiting intelligent behavior.
    \item[LBL (Lebensbegleitendes Lernen)] A concept referring to education that accompanies an individual throughout life, emphasizing continuous personal and professional development.
    \item[LLL (Live-long Learning)] A term describing the continuous process of acquiring knowledge and skills over the span of one’s lifetime.
    \item[LLMs (Large Language Models)] Advanced deep learning models designed to generate and understand human-like text based on given prompts.
    \item[MA (Magistra Artium)] An academic degree equivalent to a Master of Arts, as used in the context of Eva-Maria Egger.
    \item[Mag. (Magister/Magistra)] An academic title awarded to individuals who have completed Master’s level studies, as seen in the works of Albert Greinöcker, Eva-Maria Egger, Elke Peuschler, and Michael Prandstetter.
    \item[MIT (Massachusetts Institute of Technology)] A prestigious research university in the United States, referenced here in relation to open-source licensing contexts.
    \item[MMag.a (Magistra/Magister der Rechte)] An academic title in the field of law, equivalent to a Master of Laws, as used in the context of Eva-Maria Egger.
    \item[NLP (Natural Language Processing)] A subfield of artificial intelligence focused on the interaction between computers and human language.
    \item[OCR (Optical Character Recognition)] Technology that converts various forms of documents, such as scanned images of printed or handwritten text, into machine-encoded text.
    \item[OS (Operating System)] Software that manages computer hardware and software resources while providing common services for computer programs.
    \item[RAG (Retrieval-Augmented Generation)] An approach that combines information retrieval techniques with generative models to produce contextually enriched outputs.
    \item[RAM (Random-Access Memory)] A type of computer memory that allows data to be read or written in nearly the same amount of time irrespective of its physical location.
    \item[ROUGE (Recall-Oriented Understudy for Gisting Evaluation)] A set of metrics used to evaluate the quality of machine-generated summaries by comparing them with human-produced summaries.
    \item[SAIPiA (Earlier Project Concept)] Presumably an early project concept referenced in the time protocol, indicating preliminary or exploratory work.
    \item[SW (Software)] The collection of programs, procedures, and routines associated with the operation of a computer system.
    \item[TTS (Text-to-Speech)] Technology that converts written text into synthesized speech, enabling auditory representation of textual information.
    \item[TSN (Abbreviation Specific to HTL Anichstraße)] An abbreviation used in the context of HTL Anichstraße, as referenced in section 9.4.7.
    \item[UI (User Interface)] The means through which users interact with a computer system or software application, encompassing both hardware and software elements.
    \item[vs (versus)] A term used to indicate opposition or comparison, as seen in the title of Chapter 10.
\end{description}

