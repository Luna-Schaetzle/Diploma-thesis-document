\chapter{Glossary}
\addcontentsline{toc}{chapter}{Glossary}
\markboth{Glossary}{Glossary}
\label{glossary}
\thispagestyle{plain}

\setcounter{page}{1}
\pagenumbering{roman}

This section provides definitions and explanations for all the terms used throughout this document.

\begin{description}[leftmargin=!,labelwidth=\widthof{\bfseries REST}]
    \item[API (Application Programming Interface)] A set of protocols, tools, and definitions that allow different software applications to communicate with each other.
    \item[AI (Artificial Intelligence)] A branch of computer science focused on building systems that can perform tasks requiring human intelligence, such as learning, reasoning, and problem-solving.
    \item[IDE (Integrated Development Environment)] A software application that provides comprehensive facilities for software development, including a code editor, compiler, debugger, and automation tools.
    \item[OS (Operating System)] A system software that manages computer hardware, software resources, and provides services for computer programs.
    \item[VS Code (Visual Studio Code)] A lightweight yet powerful source code editor developed by Microsoft, supporting multiple programming languages and extensions.
    \item[REST (Representational State Transfer)] An architectural style for designing networked applications, which relies on stateless communication and standard HTTP methods.
    \item[HTTP (Hypertext Transfer Protocol)] A protocol used for communication between web browsers and servers, enabling the retrieval of hypertext documents on the internet.
    \item[JSON (JavaScript Object Notation)] A lightweight data interchange format that is easy for humans to read and write and easy for machines to parse and generate.
    \item[XML (Extensible Markup Language)] A flexible text format used to store and transport data, often used in web services and configuration files.
    \item[HTML (Hypertext Markup Language)] The standard markup language used for creating web pages and web applications.
    \item[CSS (Cascading Style Sheets)] A stylesheet language used to control the presentation and layout of web documents.
    \item[JS (JavaScript)] A programming language commonly used in web development to create interactive and dynamic web pages.
    \item[TS (TypeScript)] A strongly typed programming language that builds on JavaScript, adding static typing for improved maintainability.
    \item[SQL (Structured Query Language)] A domain-specific language used for managing and manipulating relational databases.
\end{description}
