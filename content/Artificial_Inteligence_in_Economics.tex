\chapter{Artificial Intelligence in Economics}
\label{chap:Artificial_Intelligence_in_Economics}
\textbf{Author:} Florian Prandstetter
\textbf{Author:} Luna Schaetzle


\section{Introduction}

In this chapter, we explore the role of artificial intelligence in economics and its impact on various sectors of the economy. We discuss the potential benefits and challenges of integrating AI technologies into economic processes, as well as the implications for businesses, consumers, and policymakers. Furthermore, we examine the current applications of AI in economics and highlight key trends that are shaping the future of this field.


\section{The Role of AI in Economics}

AI has revolutionized many fields of Economic. It offers a wide range of tools that can be implemented into a companys workflow.
AI can be used to optimize production processes, improve supply chain management, and enhance customer service. It can also help businesses make more informed decisions by analyzing large amounts of data and identifying patterns and trends that would be difficult for humans to detect. In addition, AI can be used to automate repetitive tasks, freeing up employees to focus on more strategic activities.

\section{Risks of AI}

While the various benefits of AI seem promising, there are also risks associated with its implementation in economics.

One of the main concerns is the potential for job displacement, as AI technologies have the potential to automate many tasks that are currently performed by humans. This could lead to widespread unemployment and economic instability if not managed properly. 

Additionally, there are ethical concerns related to the use of AI in economics, such as bias in algorithms and the potential for misuse of personal data.

It is essential for businesses and policymakers to address these risks and develop strategies to mitigate them effectively.

\cite{AiEconomics}

\section{Benefits of AI use}

But despite the riskst the benefits of AI are numerous. It can help buisness to increase their efficency and productivity.

\cite{AiEconomics2}
\cite{AiBenefits}

\subsection{Data Analyisis}

One of the most significant benefits of AI in economics is its ability to analyze large amounts of data quickly and accurately. 
This can help businesses identify trends and patterns that would be difficult for humans to detect, allowing them to make more informed decisions.

\cite{AiDataAnalysis}

\subsection{Automation}

AI can also be used to automate repetitive tasks, thus freeing up employess to focus on more important activities.
This can help businesses increase their productivity and reduce costs.

\cite{AiAutomation}

\subsection{Ressource allocation}

In sectors like manufacturing and agriculture, AI can asist in an optimal distribution of the ressources. The amount of waste reduced and the cost efficeny could be increased significantly.



\section{Applications of AI }


\subsection{Customer Service}

\subsection{Supply Chain Management}

\subsection{Predictive Analytics}



\section{Conclusion}

%so bissl zahlenwerte und so wären gut
%https://www.marketsandmarkets.com/Market-Reports/artificial-intelligence-market-74851580.html
%https://www.grandviewresearch.com/industry-analysis/artificial-intelligence-ai-market
%https://www.statista.com/statistics/607716/worldwide-artificial-intelligence-market-revenues/
