\chapter{Artificial Intelligence in Economics}
\label{chap:Artificial_Intelligence_in_Economics}
\textbf{Author:} Florian Prandstetter
\textbf{Author:} Luna Schaetzle


\section{Introduction}

In this chapter, we explore the role of artificial intelligence in economics and its impact on various sectors of the economy. We discuss the potential benefits and challenges of integrating AI technologies into economic processes, as well as the implications for businesses, consumers, and policymakers. Furthermore, we examine the current applications of AI in economics and highlight key trends that are shaping the future of this field.


\section{The Role of AI in Economics}

AI has revolutionized many fields of Economic. It offers a wide range of tools that can be implemented into a companys workflow.
AI can be used to optimize production processes, improve supply chain management, and enhance customer service. It can also help businesses make more informed decisions by analyzing large amounts of data and identifying patterns and trends that would be difficult for humans to detect. In addition, AI can be used to automate repetitive tasks, freeing up employees to focus on more strategic activities.

\section{Risks of AI}

While the various benefits of AI seem promising, there are also risks associated with its implementation in economics.

One of the main concerns is the potential for job displacement, as AI technologies have the potential to automate many tasks that are currently performed by humans. This could lead to widespread unemployment and economic instability if not managed properly. 

Additionally, there are ethical concerns related to the use of AI in economics, such as bias in algorithms and the potential for misuse of personal data.

It is essential for businesses and policymakers to address these risks and develop strategies to mitigate them effectively.

\cite{AiEconomics}

\section{Benefits of AI use}

But despite the riskst the benefits of AI are numerous. It can help buisness to increase their efficency and productivity.

\cite{AiEconomics2}
\cite{AiBenefits}

\subsection{Data Analyisis}

One of the most significant benefits of AI in economics is its ability to analyze large amounts of data quickly and accurately. 
This can help businesses identify trends and patterns that would be difficult for humans to detect, allowing them to make more informed decisions.

\cite{AiDataAnalysis}

\subsection{Automation}

AI can also be used to automate repetitive tasks, thus freeing up employess to focus on more important activities.
This can help businesses increase their productivity and reduce costs.

\cite{AiAutomation}

\subsection{Ressource allocation}

In sectors like manufacturing and agriculture, AI can asist in an optimal distribution of the ressources. The amount of waste reduced and the cost efficeny could be increased significantly.


\section{Applications of AI }


\subsection{Customer Service}

\subsection{Supply Chain Management}

\subsection{Predictive Analytics}


\section{Future Trends in AI Economics}


\section{Ethical and Legal Implications of AI in Economics}

Descripe the ethical and legal implications of AI in economics.

\subsection{Bias in Algorithms}

\subsection{Training Data Issues}

\subsection{Privacy Concerns}

\subsection{Regulatory Challenges}

\subsection{Regulatory Challenges}

The deployment of artificial intelligence in economic sectors introduces a multitude of regulatory challenges, particularly concerning data security, privacy, and compliance with both established legal frameworks and emerging regulatory measures. A landmark initiative addressing these issues is the EU AI Act (Regulation (EU) 2024/1689 \cite{EU-AI-Act-text}), which stands as the first comprehensive legal framework for artificial intelligence on a global scale. This act is designed to harmonize AI regulations across EU member states by adopting a risk-based approach that categorizes AI systems according to their potential impact on fundamental rights and public safety.

Under the EU AI Act, AI systems are classified into several risk tiers—from minimal risk to high risk—with the highest risk applications subject to stringent requirements such as mandatory conformity assessments, robust data governance practices, and enforced human oversight. These measures are intended to enhance transparency, accountability, and public trust in AI technologies. Additionally, the framework promotes continuous post-market monitoring and adaptive regulatory responses to keep pace with rapid technological advancements.

Furthermore, by establishing clear guidelines for AI deployment, the framework seeks to balance the dual objectives of fostering innovation and safeguarding societal values. This comprehensive approach not only addresses the intricate legal and ethical issues inherent in AI but also sets a precedent for future regulatory efforts at the international level \cite{EURegFrameworkAI}.

As in the chapter ~\ref{chap:Outlook} in the section ~\ref{ref:challanges-and-opportunities} disscussed there are numerous experts
that recoment to implement more and worldwide regualtions to minimize the Risk of AI.


\subsection{Data Security}

As discussed in Chapter~\ref{chap:Introduction_to_the_used_large_language_Models}, particularly in Section~\ref{sec:data-security-privacy-openai} on Data Security and Privacy in Compliance with Austrian and EU Regulations, several regulatory challenges affect the data security and privacy of AI models.

A key issue is that user data is frequently stored on cloud servers, which increases the risk of security breaches and data leaks. Moreover, many leading AI companies are headquartered outside the European Union, potentially complicating compliance with EU and Austrian data protection laws.

\section{Conclusion}

%so bissl zahlenwerte und so wären gut
%https://www.marketsandmarkets.com/Market-Reports/artificial-intelligence-market-74851580.html
%https://www.grandviewresearch.com/industry-analysis/artificial-intelligence-ai-market
%https://www.statista.com/statistics/607716/worldwide-artificial-intelligence-market-revenues/
