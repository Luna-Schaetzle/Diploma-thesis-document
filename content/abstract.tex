%%%% Time-stamp: <2013-02-25 10:31:01 vk>


\small{\chapter*{Abstract / Kurzfassung}}
\label{cha:abstract}

\small{
\textbf{Abstract:}

This diploma thesis examines the integration of Artificial Intelligence (AI) in educational and software development environments. Recognizing the increasing role of AI in automating processes, optimizing operations, and enhancing quality, as well as supporting personalized learning, the thesis addresses challenges such as complexity, limited expertise, and high development costs.
    
The primary goal was to develop AI-driven solutions for both fields. This involved setting up a server (API) to host AI models, developing a backend for an AI Hub and a Visual Studio Code Extension, and integrating suitable AI models. The work documents a shift from a broad evaluation of AI models to practical, focused applications.
    
A key aspect of the thesis is the implementation of Large Language Models (LLMs). Various free and commercial LLMs were evaluated using the Ollama application for local deployment and the OpenAI API, with quantitative and qualitative methods assessing response time, resource utilization, and text quality. This led to the selection and integration of specific models.
    
Additionally, a self-hosted Flask service was developed to link the frontend and backend components for both the Intelligent Student AI Hub and the Visual Studio Code Extension. The thesis also outlines the architecture of a web platform featuring an interactive chatbot, image-to-text tool, programming bot, and user account management, and includes economic evaluations of AI and open-source technologies. The project’s source code is available on GitHub under the GPL-3.0 License.
    
Key findings highlight the transformative potential of AI in education and industry, and although some planned features were omitted due to time constraints, the modular architecture supports future enhancements.

\newpage

\textbf{Kurzfassung:}
    
Diese Diplomarbeit untersucht die Integration Künstlicher Intelligenz (KI) in Bildungs- und Softwareentwicklungsumgebungen. Angesichts der zunehmenden Bedeutung von KI für Prozessautomatisierung, Optimierung und Qualitätssteigerung sowie zur Unterstützung personalisierten Lernens werden Herausforderungen wie Komplexität, Fachkräftemangel und hohe Entwicklungskosten beleuchtet.
    
Ziel war die Entwicklung KI-gestützter Lösungen für beide Bereiche, wozu die Einrichtung eines Servers (API) zum Hosten von KI-Modellen, die Entwicklung eines Backends für einen KI-Hub und eine Visual Studio Code Extension sowie die Integration geeigneter KI-Modelle gehören. Die Arbeit beschreibt den Übergang von einer breiten Evaluierung zu praxisnahen Anwendungen.
    
Ein Schwerpunkt liegt auf der Implementierung von Large Language Models (LLMs), die mittels der Ollama-Applikation und der OpenAI API bewertet wurden. Quantitative und qualitative Methoden zur Bewertung von Ressourcenauslastung, Antwortzeit und Textqualität führten zur Integration spezifischer Modelle.
    
Zudem wurde ein selbst gehosteter Flask-Dienst entwickelt, der als Schnittstelle zwischen Frontend und Backend für den Student AI Hub und die Visual Studio Code Extension dient. Die Architektur der Webplattform – mit interaktivem Chatbot, Bild-zu-Text-Werkzeug, Programmierbot und Benutzerverwaltung – sowie wirtschaftliche Betrachtungen von KI- und Open-Source-Technologien runden die Arbeit ab. Der Quellcode ist auf GitHub unter der GPL-3.0 Lizenz verfügbar.
    
Die Ergebnisse unterstreichen das transformative Potenzial von KI in Bildung und Industrie und bilden eine Basis für zukünftige Erweiterungen.    
}