\chapter{Operating Systems}
\label{chap:Operating_Systems_used}

% make some introduction
% add the os that were evaluated for the project
% write the outcome of the evaluation
% write the reason why the os was chosen
% write the reason why the os was not chosen
% Maby about Raspberry Pi OS (Maby in Gabis Part)



\section{What is an Operating System?}

An Operating System (OS) acts as an intermediary between the user and the computer hardware, ensuring smooth and efficient operation of the system. It is responsible for managing hardware components such as the CPU, memory, storage devices, and input/output peripherals, allowing users and applications to interact with them seamlessly.

The OS provides essential functions like process management ensuring that multiple applications can run without conflicts. It also handles memory management by allocating and deallocating memory to different processes. It also prevents unauthorized access to the storage.
Additonally OS handles file system management, enabling organized and efficient data storage. 

Another key role of the OS is device management, where it communicates with hardware components using device drivers.

Furthermore Operating Systems provide networking capabilities enabling connection to local and global networks. They try to ensure a stable and secure connection between devices.


\cite{WhatIsAnOs}

\subsection {Types of Operating Systems}

The type of Operating System used is determined by the needs of the user. 
Different tpyes of Operating Systems use different system architectures to provide varieing benefits.
Factors like scalability, performance and reliability have to be considered in choosing a suitable os.


\begin{itemize}

    \item \textbf{Batch Operating Systems} are designed to handle a large number of processes. They are used to process large amounts of data and to run complex calculations.
    \begin{itemize}
        \item \textbf{Advantages} multiple users can share the batch system and it is easy to manage big amounts of traffic

        \item \textbf{Disadvantages} The CPU isn't used efficently. Also the response time can be slow, due to processes being processed one by one.
    
        
    
    \end{itemize} 

    %maybe more detailed
    \item \textbf{Multi Programming/Time Sharing Operating System} are used to utilize the available ressources as efficent as possible. They allow using multiple programms at the same time, which allow multiple users to share the system.
    They are often used for servers and also have been used by the devolpmentt team in the project.

    \begin{itemize}
        \item \textbf{Advantages} Allows multiple users to share the ressources. Alos it helps avoiding duplicated software.
        \item \textbf{Disadvantages} There can be problems with reliable data communication. 
    \end{itemize} 


    \item \textbf{Real Time Operating System} are used when a very short response time is needed. They are often used on microcontrollers like the ESP32.
    \begin{itemize}
        \item \textbf{Advantages} They utilize the device to it's maximum. Fast and reliable memory allocation. They usally are error free.
        \item \textbf{Disadvantages} There can omly be a limited amount of tasks. They also require complex algorithms that can be resource heavy.
    \end{itemize} 
    

\end{itemize}   

\cite{TypesOfOs}

\section {Operating Systems used}

\subsection {AI Server}

\subsection


\author{Florian Prandstetter}