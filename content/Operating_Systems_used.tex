\chapter{Operating Systems (Florian Prandstetter)}
\label{chap:Operating_Systems_used}

% make some introduction
% add the os that were evaluated for the project
% write the outcome of the evaluation
% write the reason why the os was chosen
% write the reason why the os was not chosen
% Maybe about Raspberry Pi OS (Maybe in Gabi's Part)

\section {Introduction}

An important part of building an AI Service is to host an API endpoint for all applications to use. To improve the quality of the service, having a flawless and seamless connection with a fast response time is necessary.
Having a good foundation to build upon is thus unavoidable to build a strong and secure API endpoint. 

This chapter is going to briefly explain what an Operating System (OS) is and its importance.
It will also evaluate different options suitable for hosting the AI Server.

\section{What is an Operating System?}
\label{sec:WhatIsAnOs}

An Operating System (OS) acts as an intermediary between the user and the computer hardware, ensuring smooth and efficient operation of the system. It is responsible for managing hardware components such as the CPU, memory, storage devices, and input/output peripherals, allowing users and applications to interact with them seamlessly.

The OS provides essential functions like process management, ensuring that multiple applications can run without conflicts. It also handles memory management by allocating and deallocating memory to different processes. It also prevents unauthorized access to the storage.
Additionally, the OS handles file system management, enabling organized and efficient data storage. 

Another key role of the OS is device management, where it communicates with hardware components using device drivers.

Furthermore, Operating Systems provide networking capabilities, enabling connection to local and global networks. They try to ensure a stable and secure connection between devices.

\cite{WhatIsAnOs}

\subsection {Types of Operating Systems}

The type of Operating System used is determined by the needs of the user. 
Different types of Operating Systems use different system architectures to provide varying benefits.
Factors like scalability, performance, and reliability have to be considered in choosing a suitable OS.

\begin{itemize}

    \item \textbf{Batch Operating Systems} are designed to handle a large number of processes. They are used to process large amounts of data and to run complex calculations.
    \begin{itemize}
        \item \textbf{Advantages} Multiple users can share the batch system, and it is easy to manage large amounts of traffic.
        \item \textbf{Disadvantages} The CPU isn't used efficiently. Also, the response time can be slow due to processes being processed one by one.
    \end{itemize} 

    \item \textbf{Multi Programming/Time Sharing Operating System} is used to utilize the available resources as efficiently as possible. They allow using multiple programs at the same time, which allows multiple users to share the system.
    They are often used for servers and also have been used by the development team in the project.

    \begin{itemize}
        \item \textbf{Advantages} Allows multiple users to share the resources. Also, it helps avoid duplicated software.
        \item \textbf{Disadvantages} There can be problems with reliable data communication. 
    \end{itemize} 

    \item \textbf{Real-Time Operating System} is used when a very short response time is needed. They are often used on microcontrollers like the ESP32.
    \begin{itemize}
        \item \textbf{Advantages} They utilize the device to its maximum. Fast and reliable memory allocation. They usually are error-free.
        \item \textbf{Disadvantages} There can only be a limited number of tasks. They also require complex algorithms that can be resource-heavy.
    \end{itemize} 

\end{itemize}   

\cite{TypesOfOs}

\section {Operating Systems used on the server}

To host the required AI service, choosing a suitable Operating System is a crucial part to guarantee a satisfying performance.
For the project, multiple users need to access the server simultaneously, so the development team decided to use a Multi Programming OS.

\subsection {Evaluating different Operating Systems}

The preferred OS for an AI Server is Linux distributions like Ubuntu or Red Hat.
Their open-source nature makes it very easy to customize and integrate AI frameworks.

The Operating Systems that were taken into consideration are Ubuntu Server, Debian, and Red Hat Enterprise. 

\cite{LinuxPoweredAi}

\subsubsection{Debian}

Debian is a good choice for servers due to its stability and security. It is widely used since it has a large package repository and offers long-term support. It also has a large community that provides good documentation and resources. The distribution is well-suited for hosting AI services and provides a good foundation for building and deploying applications.

\cite{LinuxForServerDebian}

\subsubsection{Red Hat Enterprise}

Red Hat Enterprise is a commercial Linux distribution. It is the most popular distribution for servers. It comes with a lot of features that are useful for hosting AI services. It is a good choice for companies that need a stable and secure server. It also provides long-term support and has good documentation.
The main downside is that it is a commercial distribution and requires a subscription.

\cite{LinuxForServerRedhead}

\subsubsection{Ubuntu Server}

Ubuntu Server is a popular choice for hosting AI services. It is based on Debian and has a large package repository. It is easy to use and has a large community that provides good documentation and support. It also provides NVIDIA drivers and CUDA support, which is useful for running AI applications that require GPU acceleration.

\cite{LinuxForServerUbuntu} 

\subsection {Outcome of the Evaluation}

After evaluating the different Operating Systems, the development team decided to use Ubuntu Server for hosting the AI Service.
The main reason for the decision was the experience of the team with Ubuntu and the easy integration of AI frameworks like TensorFlow and PyTorch.
Also, the large community and the good documentation were important factors in the decision. The NVIDIA drivers were also an important factor that played a role in the decision.


\section{Raspberry Pi Operating System(Gabiriel ???)}

\subsection{Raspberry Pi OS}


\author{Florian Prandstetter}
