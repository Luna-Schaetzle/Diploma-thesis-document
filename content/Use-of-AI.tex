\chapter{Information zur Verwendung von Künstlicher Intelligenz //| Information on the Use of Artificial Intelligence}
\label{cha:Use-of-AI}

\section*{Deutsch}
Die vorliegende Diplomarbeit nutzt Künstliche Intelligenz (KI) als integralen Bestandteil des Forschungs- und Schreibprozesses. Ziel war es, die wissenschaftliche Arbeit durch den Einsatz moderner Technologien zu unterstützen, ohne die intellektuelle Eigenleistung und kritische Bewertung der Autorin bzw. des Autors zu ersetzen. Dabei wurde großer Wert darauf gelegt, sämtliche rechtlichen, ethischen und methodischen Aspekte vollumfänglich zu berücksichtigen.

\subsection*{Anwendungsbereiche und Einsatz der KI}
Im Rahmen dieser Arbeit kamen verschiedene KI-basierte Tools zum Einsatz:
\begin{itemize}
    \item \textbf{Textgenerierung und Sprachliche Optimierung:} Unterstützung bei der Formulierung, Strukturierung und stilistischen Verfeinerung wissenschaftlicher Inhalte. 
    \item \textbf{Datenanalyse:} Verarbeitung und Analyse großer Datenmengen zur Identifikation relevanter Muster und Erkenntnisse.
    \item \textbf{Feedback und Verbesserungsvorschläge:} Generierung konstruktiver Hinweise zur inhaltlichen und methodischen Optimierung.
\end{itemize}

\newpage

Zu den in diesem Projekt verwendeten KI-Tools zählen unter anderem:
\begin{itemize}
    \item ChatGPT
    \item GitHub Copilot
    \item Deepseek
    \item Ollama Modells
    \item Mistrals le Chat
\end{itemize}

\subsection*{Wissenschaftliche Methodik und Validierung}
Alle durch KI generierten Inhalte und Vorschläge wurden einer sorgfältigen wissenschaftlichen Überprüfung unterzogen. Die methodische Vorgehensweise umfasste:
\begin{itemize}
    \item Eine manuelle Prüfung aller KI-Ergebnisse hinsichtlich inhaltlicher Richtigkeit und Relevanz.
    \item Die Integration der KI-bezogenen Ergebnisse in den Gesamtzusammenhang der Arbeit unter strenger Beachtung der etablierten wissenschaftlichen Qualitätsstandards.
    \item Eine kritische Reflexion der Limitationen der verwendeten KI-Methoden, um Verzerrungen und fehlerhafte Interpretationen zu vermeiden.
\end{itemize}
So wurde sichergestellt, dass die KI-gestützten Beiträge ausschließlich als ergänzende Hilfsmittel zur Steigerung der Effizienz und Qualität der Arbeit verwendet wurden.

\subsection*{Rechtliche Absicherung und Haftungsausschluss}
Die Nutzung der genannten KI-Tools erfolgte unter strikter Einhaltung aller geltenden gesetzlichen Bestimmungen. Hierzu zählen insbesondere:
\begin{itemize}
    \item Die Einhaltung der Datenschutzbestimmungen, insbesondere im Hinblick auf die Verarbeitung personenbezogener Daten.
    \item Die Berücksichtigung der spezifischen Nutzungsbedingungen und vertraglichen Vereinbarungen der jeweiligen Anbieter.
\end{itemize}
Es wird ausdrücklich darauf hingewiesen, dass die alleinige Verantwortung für die wissenschaftliche Integrität und die inhaltliche Richtigkeit dieser Arbeit bei der Autorin bzw. dem Autor liegt. Die von den KI-Tools generierten Ergebnisse sind als unterstützende Instrumente zu verstehen und stellen keine eigenständigen wissenschaftlichen Erkenntnisse dar. Alle automatisiert erzeugten Inhalte wurden manuell überprüft und bei Bedarf korrigiert. Mit der Einreichung dieser Arbeit bestätigt die Autorin bzw. der Autor, dass sämtliche rechtlichen, ethischen und methodischen Rahmenbedingungen eingehalten wurden.

\subsection*{Ethische Überlegungen und Transparenz}
Neben der Einhaltung rechtlicher Vorgaben wurde auch auf die ethische Dimension des KI-Einsatzes geachtet. Transparenz im Umgang mit den eingesetzten Technologien war ein zentraler Aspekt dieser Arbeit:
\begin{itemize}
    \item Es wurde offen gelegt, welche KI-Tools zum Einsatz kamen und in welchen Bereichen sie genutzt wurden.
    \item Die Limitationen der KI-gestützten Verfahren wurden kritisch reflektiert, um eine einseitige Darstellung der wissenschaftlichen Ergebnisse zu vermeiden.
    \item Der Einsatz der KI diente ausschließlich der Unterstützung und nicht der Substitution der eigenständigen wissenschaftlichen Arbeit.
\end{itemize}

\section*{English}
This diploma thesis integrates Artificial Intelligence (AI) as a fundamental component of the research and writing process. The primary objective was to enhance the quality and efficiency of the academic work through modern technological support, while ensuring that the intellectual contribution and critical oversight of the author remain paramount. Special care was taken to fully comply with all relevant legal, ethical, and methodological requirements.

\subsection*{Application Areas and Use of AI}
In this work, various AI-based tools were employed:
\begin{itemize}
    \item \textbf{Text Generation and Linguistic Optimization:} Assist in formulating, structuring, and refining the stylistic aspects of academic content.
    \item \textbf{Data Analysis:} Process and analyze large datasets to identify significant patterns and insights.
    \item \textbf{Feedback and Improvement Suggestions:} Generate constructive recommendations for content and methodological enhancements.
\end{itemize}
The AI tools used in this project include, but are not limited to:
\begin{itemize}
    \item ChatGPT
    \item GitHub Copilot
    \item Deepseek
    \item Ollama Modells
    \item Mistrals le Chat
\end{itemize}

\subsection*{Scientific Methodology and Validation}
All outputs and suggestions generated by AI were subjected to rigorous scientific review. The methodological approach included:
\begin{itemize}
    \item A thorough manual review of all AI-generated content to ensure its accuracy and relevance.
    \item Integrating the AI-assisted results within the broader context of the thesis while strictly adhering to established scientific quality standards.
    \item Critically reflecting on the limitations of the AI methods employed to avoid biases and erroneous interpretations.
\end{itemize}
This process ensured that AI-based contributions were used solely as auxiliary tools to enhance the efficiency and quality of the work.

\subsection*{Legal Safeguards and Disclaimer}
The use of the aforementioned AI tools was carried out in strict accordance with all applicable legal requirements. In particular, the following measures were observed:
\begin{itemize}
    \item Adherence to data protection laws, especially regarding the processing of personal data.
    \item Consideration of the specific terms of service and contractual agreements of the respective providers.
\end{itemize}
It is expressly stated that the sole responsibility for the scientific integrity and accuracy of this work rests with the author. The outputs generated by the AI tools are to be regarded as supplementary aids and do not constitute independent scientific findings. All automated outputs were manually verified and corrected if necessary. By submitting this work, the author confirms that all legal, ethical, and methodological frameworks have been fully observed.

\subsection*{Ethical Considerations and Transparency}
In addition to legal compliance, significant emphasis was placed on the ethical aspects of using Artificial Intelligence. Transparency regarding the employed technologies was a key element of this work:
\begin{itemize}
    \item Full disclosure of the AI tools used and the areas in which they were applied.
    \item A critical reflection on the limitations of the AI-assisted methods to prevent biased or incomplete representation of the scientific results.
    \item Ensuring that the use of AI served solely as a supportive tool and did not replace the author’s independent scholarly work.
\end{itemize}
